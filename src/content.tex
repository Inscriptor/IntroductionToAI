% Title page
\frame[plain]{\titlepage}

\lecture{Введение}{intro}

\section{Факты, правила и запросы}
\subsection{Основные пункты}

\begin{frame}
	\frametitle{\insertsection}
	\framesubtitle{\insertsubsection}
	\begin{itemize}
		\item Рассмотреть простые примеры программ на \textbf{Prolog}, определить понятия факта, правила, запроса и базы знаний
		\item Дать определения основных понятий и синтаксических конструкций, таких как атомы, переменные и термы
	\end{itemize}
\end{frame}

\subsection{База знаний}

\begin{frame}
	\frametitle{\insertsection}
	\framesubtitle{\insertsubsection}
	Базовые конструкции языка \textbf{Prolog}:
	\begin{itemize}
		\item Факты
		\item Правила
		\item Запросы
	\end{itemize}
	Программа на Prolog представляет собой множество \textbf{фактов} и \textbf{правил}~--- \alert{Базу Знаний}
	Чтобы использовать информацию, содержащуюся в базе знаний, необходимо ставить \textbf{запросы}.
\end{frame}

\begin{frame}
	\frametitle{\insertsection}
	\framesubtitle{\insertsubsection}
	Пример 1
	
	
	\texttt{\begin{itemize}
				\item[] woman(mia).
				\item[] woman(jody).
				\item[] woman(yolanda).
				\item[] playsGuitar(jody).
				\item[]<2-> listensToMusic(mia).
				\item[]<2-> musician(yolanda).
				\item[]<2-> playsGuitar(mia) :- listensToMusic(mia).
				\item[]<2-> playsGuitar(yolanda) :- listensToMusic(yolanda).
				\item[]<2-> listensToMusic(yolanda) :- musician(yolanda).
			\end{itemize}}
\end{frame}

\begin{frame}
	\frametitle{\insertsection}
	\framesubtitle{\insertsubsection}
	Пример 2
	
	
	\texttt{\begin{itemize}
		\item[] musician(mia).
		\item[] listensToMusic(jody).
		\item[] playsGuitar(mia) :- listensToMusic(mia),musician(mia).
		\item[] playsGuitar(jody) :- musician(jody); listensToMusic(jody).
	\end{itemize}}
\end{frame}

\begin{frame}
	\frametitle{\insertsection}
	\framesubtitle{\insertsubsection}
	Пример 3
	
	
	\texttt{\begin{itemize}
		\item[] woman(mia).
		\item[] woman(jody).
		\item[] woman(yolanda).
		\item[] loves(vincent,mia).
		\item[] loves(marcellus,mia).
	\end{itemize}}
\end{frame}

\begin{frame}
	\frametitle{\insertsection}
	\framesubtitle{\insertsubsection}
	Пример 4
	
	
	\texttt{\begin{itemize}
			\item[] loves(vincent,mia).
			\item[] loves(marcellus,mia).
			\item[] jealous(X,Y) :- loves(X,Z),loves(Y,Z).
	\end{itemize}}
\end{frame}

\subsection{Синтаксис}

\begin{frame}
	\frametitle{\insertsection}
	\framesubtitle{\insertsubsection}
	\textbf{\underline{Атомы}}
	
	\begin{enumerate}
		\item Последовательность строчных или прописных букв, цифр и символов подчерка, начинающаяся со строчной буквы.
		\item Произвольная последовательность символов, заключённая в одинарные кавычки.
		\item Последовательность спецсимволов.
	\end{enumerate}

	\begin{rexample}
		mia, marcellus, big\_kahuna\_burger, 'Произвольная строка символов', ====>, :-
	\end{rexample}
\end{frame}

\begin{frame}
	\frametitle{\insertsection}
	\framesubtitle{\insertsubsection}
	\textbf{\underline{Числа}}
	
	\begin{enumerate}
		\item Действительные числа: 2,718; 103,3087; \(\pi \), \ldots
		\item Целые числа: -2, -1, 0, 1, 2, \ldots
	\end{enumerate}

	\textbf{\underline{Переменные}}
	
	\textit{Переменной} называется последовательность строчных или прописных букв, цифр и символов подчерка, начинающаяся с \textbf{прописной буквы} или \textbf{символа подчерка}.
	
	\begin{rexample}
		X, Y, Variable, \_X, X1, \_variable\_with\_some\_info\_
	\end{rexample}
\end{frame}

\begin{frame}
	\frametitle{\insertsection}
	\framesubtitle{\insertsubsection}
	\textbf{\underline{Термы}}
	
	\textit{Терм (составной терм)} состоит из \alert{функтора} и последовательности аргументов в скобках.
	\begin{enumerate}
		\item Любой атом или число является термом. Такие термы называются \alert{константами}.
		\item Любая переменная является термом.
		\item Имя функтора~--- это атом.
		\item Переменная не может быть функтором.
		\item Аргументы составного терма должны быть термами.
	\end{enumerate}

	\begin{rexample}
		loves(vincent, mia), playsGuitar(jody), jody, musician(mia), eats(cat,Prey)
	\end{rexample}
\end{frame}

\subsection{Проверочные вопросы}

\begin{frame}
	\frametitle{\insertsection}
	\framesubtitle{\insertsubsection}
	Какие из перечисленных строк являются атомами, какие переменными, а какие не являются ни тем, ни другим?
	\texttt{\begin{enumerate}
		\item vINCENT
		\item Foot
		\item x1
		\item Y3
		\item big\_kahuna\_burger
		\item 'Криминальное чтиво'
		\item roast chicken
		\item \_IndianaJones
		\item '\_IndianaJones'
	\end{enumerate}}
\end{frame}

\begin{frame}
	\frametitle{\insertsection}
	\framesubtitle{\insertsubsection}
	Какие из перечисленных ниже строк являются атомами, переменными или составными термами, а какие вообще не являются термами? Для каждого составного терма укажите имя функтора и его арность.
	\texttt{\begin{enumerate}
		\item loves(vincent,mia)
		\item 'loves(vincent,mia)'
		\item Eats(cat,mouse)
		\item hasChildren(cat,kittens)
		\item and(musician(jody),artist(mia))
		\item and(musician(X),artist(Y))
		\item \_and(musician(jody),artist(mia))
		\item (Butch kills Vincent)
		\item kills(Butch,Vincent)
		\item kills(Butch,Vincent
	\end{enumerate}}
\end{frame}

\begin{frame}
	\frametitle{\insertsection}
	\framesubtitle{\insertsubsection}
	Сколько фактов, правил, высказываний и предикатов в следующей базе знаний? Для каждого правила назовите вывод и цели.
	\texttt{\begin{itemize}
			\item[] woman(mia).
			\item[] woman(jody).
			\item[] man(jules).
			\item[] person(X) :- man(X); woman(X).
			\item[] loves(X,Y) :- knows(Y,X).
			\item[] father(Y,Z) :- man(Y), son(Z,Y).
			\item[] father(Y,Z) :- man(Y), daughter(Z,Y).
	\end{itemize}}
\end{frame}

\subsection{Упражнения}

\begin{frame}
	\frametitle{\insertsection}
	\framesubtitle{\insertsubsection}
	Запишите следующую базу знаний на языке Prolog.
	
	\begin{itemize}
		\item Бутч убийца.
		\item Миа и Марселлас женаты.
		\item Зед мертв.
		\item Марселлас убьет любого, кто сделает Мие массаж стопы.
		\item Миа любит любого, кто хорошо танцует.
		\item Джулс ест все, что вкусно или питательно.
	\end{itemize}
\end{frame}


\lecture{Matching}{match}

\section{Согласование термов и поиск решений}
\subsection{Основные пункты}

\begin{frame}
	\frametitle{\insertsection}
	\framesubtitle{\insertsubsection}
	\begin{itemize}
		\item Познакомиться с понятием согласования термов.
		\item Разобраться в стратегиях, используемых движком языка Prolog для поиска ответов на вопросы.
	\end{itemize}
\end{frame}